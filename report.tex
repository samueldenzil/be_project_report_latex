\documentclass[12pt]{article}
\usepackage{blindtext}
\usepackage{fancyhdr}
\usepackage{indentfirst}
\usepackage{geometry}
\geometry{
    a4paper,
    total={170mm,257mm},
    left=37mm,
    top=25.4mm,
    bottom=32mm,
    right=25.4mm
}
\setlength{\parindent}{2cm}
\begin{document}
% \maketitle
\tableofcontents
% Header and footer
\pagestyle{fancy}
\fancyhead{} % clear all header fields
\fancyhead[C]{3D to 2D Floor Plan Modelling using Image Processing and Augmented Reality}
\fancyfoot{} % clear all footer fields
\fancyfoot[C]{\footnotesize{PVG’s College of Engineering and Technology and G. K. Pate (Wani) Institute of Management Information Technology 2022-23}}

\section{\large{INTRODUCTION}}
Augmented reality is a field that combines some merging technologies in computer science, including image processing, vision graphics, human-computer interaction, and virtual reality. Augmented reality uses computer-generated virtual information to mix with what users observe in the real environment. Today’s technological advances have caused many businesses to take advantage of virtual reality and augmented reality technology to attract customer attention, such as in simulation, maintenance, marketing and engineering. Many methods can be used to implement augmented reality, one of them is by utilizing homography and feature extraction to search for predefined markers and draw a model on them.

\section{\large{ABSTRACT}}
Smartphones currently support the augmented reality feature, capable of displaying virtual objects into the real-world environment at a specific location designated with a marker. This research develops an augmented reality application on a smartphone to visualize a 3D house design. The floorplan image captured by the smartphone camera is designated as a marker and then sent to the server. The server-side application, built using deep learning and integer programming techniques, detects corner positions on the image and produces a vector containing 2D coordinates. The client application on the smartphone uses the 2D coordinates to draw a 3D house model using Unity 3D. The model is then added over the floor plan image marker using Vuforia. Based on the test results, the 3D house models can be drawn according to the provided floor plan with moderate accuracy. The augmented reality application offers an alternative approach to examining a floor plan so that a user can better understand and engage with the design.


\section{\large{BACKGROUND AND LITERATURE SURVEY}}
The system flow is that the user first places a floorplan that will be presented in front of the smartphone’s camera. The camera then takes a floorplan image, which is then examined whether it is suitable as a marker or not. If the image is appropriate, then it is sent to the server. Otherwise, the client application will ask the user to take a more suitable image. The image should be clear with an upright position and contains enough features. Small tolerance in scaling and rotation is allowable. The algorithm was used to create an augmented reality marker.

After receiving the client’s image, the server runs the image recognition algorithm and prepares a tracking system. We adopt the learning-based approach to convert the raster plan image into a computer-readable vector representation. Then, the server sends the results of the recognition algorithm and tracking system to the client. The algorithm is used on the server side. After receiving the required information, the client draws a 3D model using the received coordinate points.\\
Steps - \\
a) Augmented reality marker\\
b) Floorplan recognition\\
c) 3D model building\\
d) Floorplan visualization\\

\section{\large{Problem Statement/definition.}}

\section{\large{Software Requirement Specification (In SRS Documentation only).}}

\section{\large{Flowchart}}

\section{\large{Project Requirement specification.}}

\section{\large{Proposed system Architecture.}}

\section{\large{High level design of the project (DFD,UML, ER Diagrams).}}

\section{\large{System implementation-code documentation: Algorithm style, Description of detailed methodologies,  protocols used etc..as applicable.}}

\section{\large{Test cases.}}

\section{\large{Proposed GUI/Working modules/Experimental Results (Module wise if available) in suitable format.}}

\section{\large{Project Plan.}}

\end{document}
